% !TEX TS-program = pdflatexmk

% This is "sig-alternate.tex" V2.1 April 2013
% This file should be compiled with V2.5 of "sig-alternate.cls" May 2012
%
% This example file demonstrates the use of the 'sig-alternate.cls'
% V2.5 LaTeX2e document class file. It is for those submitting
% articles to ACM Conference Proceedings WHO DO NOT WISH TO
% STRICTLY ADHERE TO THE SIGS (PUBS-BOARD-ENDORSED) STYLE.
% The 'sig-alternate.cls' file will produce a similar-looking,
% albeit, 'tighter' paper resulting in, invariably, fewer pages.
%
% ----------------------------------------------------------------------------------------------------------------
% This .tex file (and associated .cls V2.5) produces:
%       1) The Permission Statement
%       2) The Conference (location) Info information
%       3) The Copyright Line with ACM data
%       4) NO page numbers
%
% as against the acm_proc_article-sp.cls file which
% DOES NOT produce 1) thru' 3) above.
%
% Using 'sig-alternate.cls' you have control, however, from within
% the source .tex file, over both the CopyrightYear
% (defaulted to 200X) and the ACM Copyright Data
% (defaulted to X-XXXXX-XX-X/XX/XX).
% e.g.
% \CopyrightYear{2007} will cause 2007 to appear in the copyright line.
% \crdata{0-12345-67-8/90/12} will cause 0-12345-67-8/90/12 to appear in the copyright line.
%
% ---------------------------------------------------------------------------------------------------------------
% This .tex source is an example which *does* use
% the .bib file (from which the .bbl file % is produced).
% REMEMBER HOWEVER: After having produced the .bbl file,
% and prior to final submission, you *NEED* to 'insert'
% your .bbl file into your source .tex file so as to provide
% ONE 'self-contained' source file.
%
% ================= IF YOU HAVE QUESTIONS =======================
% Questions regarding the SIGS styles, SIGS policies and
% procedures, Conferences etc. should be sent to
% Adrienne Griscti (griscti@acm.org)
%
% Technical questions _only_ to
% Gerald Murray (murray@hq.acm.org)
% ===============================================================
%
% For tracking purposes - this is V2.0 - May 2012

\documentclass{sig-alternate-05-2015}

% For figures
\usepackage{graphicx} % more modern
%\usepackage{epsfig} % less modern
\usepackage{subfigure}

% for citation
\usepackage{natbib}

% For algorithms
\usepackage{algorithm}
\usepackage{algorithmic}

\usepackage{multirow}
%\usepackage{amsthm}

\usepackage[svgnames]{xcolor}
\usepackage{tabu}
\definecolor{navy}{rgb}{0.1, 0.1, 0.8}
\definecolor{gray}{rgb}{0.6, 0.6, 0.6}
\taburulecolor{gray}

\newcommand{\eat}[1]{}
\newcommand{\rev}[1]{{\color{black}{#1}}}
\newcommand{\verify}[1]{{\color{black}{#1}}}
\newcommand{\TODO}[2]{ {\bf [{#1}:~{#2}]}}

%spacing tricks
\newcommand{\secmoveup}{\vspace{-0.mm}} %{\vspace{-0.5mm}}
\newcommand{\bigsecmoveup}{\secmoveup\vspace{-0.0mm}}
\newcommand{\textmoveup}{\vspace{-0.mm}} %{\vspace{-0.5mm}}         	%{\vspace{-0.08in}}
\newcommand{\bigtextmoveup}{\textmoveup\vspace{-.0mm}}	%{\vspace{-0.06in}}
\newcommand{\itemmoveup}{\vspace{-0.mm}}              %{\vspace{-0.04in}}
\newcommand{\eqmoveup}{\vspace{-0.mm}}                %{\vspace{-0.16in}}
\newcommand{\captionmoveup}{\eqmoveup\vspace{-1.mm}}  %{\vspace{-0.16in}}
\newcommand{\tablemoveup}{\eqmoveup\vspace{-.0mm}}   %{\vspace{-0.16in}}


\begin{document}

% Copyright
\setcopyright{acmcopyright}
%\setcopyright{acmlicensed}
%\setcopyright{rightsretained}
%\setcopyright{usgov}
%\setcopyright{usgovmixed}
%\setcopyright{cagov}
%\setcopyright{cagovmixed}

% DOI
\doi{10.475/123_4}

% ISBN
\isbn{123-4567-24-567/08/06}

%Conference
\conferenceinfo{PLDI '13}{June 16--19, 2013, Seattle, WA, USA}

\acmPrice{\$15.00}

%
% --- Author Metadata here ---
\conferenceinfo{WOODSTOCK}{'97 El Paso, Texas USA}
%\CopyrightYear{2007} % Allows default copyright year (20XX) to be over-ridden - IF NEED BE.
%\crdata{0-12345-67-8/90/01}  % Allows default copyright data (0-89791-88-6/97/05) to be over-ridden - IF NEED BE.
% --- End of Author Metadata ---

\title{Thompson Sampling and Compositions in Knowledge Bases with Uncertainty}
%\subtitle{[Extended Abstract]
%\titlenote{A full version of this paper is available as
%\textit{Author's Guide to Preparing ACM SIG Proceedings Using
%\LaTeX$2_\epsilon$\ and BibTeX} at
%\texttt{www.acm.org/eaddress.htm}}}
%
% You need the command \numberofauthors to handle the 'placement
% and alignment' of the authors beneath the title.
%
% For aesthetic reasons, we recommend 'three authors at a time'
% i.e. three 'name/affiliation blocks' be placed beneath the title.
%
% NOTE: You are NOT restricted in how many 'rows' of
% "name/affiliations" may appear. We just ask that you restrict
% the number of 'columns' to three.
%
% Because of the available 'opening page real-estate'
% we ask you to refrain from putting more than six authors
% (two rows with three columns) beneath the article title.
% More than six makes the first-page appear very cluttered indeed.
%
% Use the \alignauthor commands to handle the names
% and affiliations for an 'aesthetic maximum' of six authors.
% Add names, affiliations, addresses for
% the seventh etc. author(s) as the argument for the
% \additionalauthors command.
% These 'additional authors' will be output/set for you
% without further effort on your part as the last section in
% the body of your article BEFORE References or any Appendices.

\numberofauthors{3} %  in this sample file, there are a *total*
% of EIGHT authors. SIX appear on the 'first-page' (for formatting
% reasons) and the remaining two appear in the \additionalauthors section.
%
\author{
% You can go ahead and credit any number of authors here,
% e.g. one 'row of three' or two rows (consisting of one row of three
% and a second row of one, two or three).
%
% The command \alignauthor (no curly braces needed) should
% precede each author name, affiliation/snail-mail address and
% e-mail address. Additionally, tag each line of
% affiliation/address with \affaddr, and tag the
% e-mail address with \email.
%
% 1st. author
\alignauthor Dongwoo Kim\\
       \affaddr{ANU}\\
       \affaddr{Australia}\\
       \email{dongwoo.kim@anu.edu.au}
% 2nd. author
\alignauthor Lexing Xie\\
       \affaddr{ANU \& NICTA}\\
       \affaddr{Australia}\\
       \email{lexing.xie@anu.edu.au}
% 3rd. author
\alignauthor  Cheng Soon Ong\\
       \affaddr{NICTA \& ANU}\\
       \affaddr{Australia}\\
       \email{chengsoon.ong@anu.edu.au}
%\and  % use '\and' if you need 'another row' of author names
%% 4th. author
%\alignauthor Lawrence P. Leipuner\\
%       \affaddr{Brookhaven Laboratories}\\
%       \affaddr{Brookhaven National Lab}\\
%       \affaddr{P.O. Box 5000}\\
%       \email{lleipuner@researchlabs.org}
%% 5th. author
%\alignauthor Sean Fogarty\\
%       \affaddr{NASA Ames Research Center}\\
%       \affaddr{Moffett Field}\\
%       \affaddr{California 94035}\\
%       \email{fogartys@amesres.org}
%% 6th. author
%\alignauthor Charles Palmer\\
%       \affaddr{Palmer Research Laboratories}\\
%       \affaddr{8600 Datapoint Drive}\\
%       \affaddr{San Antonio, Texas 78229}\\
%       \email{cpalmer@prl.com}
}
% There's nothing stopping you putting the seventh, eighth, etc.
% author on the opening page (as the 'third row') but we ask,
% for aesthetic reasons that you place these 'additional authors'
% in the \additional authors block, viz.
%\additionalauthors{Additional authors: John Smith (The Th{\o}rv{\"a}ld Group,
%email: {\texttt{jsmith@affiliation.org}}) and Julius P.~Kumquat
%(The Kumquat Consortium, email: {\texttt{jpkumquat@consortium.net}}).}
%\date{30 July 1999}
% Just remember to make sure that the TOTAL number of authors
% is the number that will appear on the first page PLUS the
% number that will appear in the \additionalauthors section.

\maketitle

%!TEX root = ./icml2016.tex
\begin{abstract}
  Automated processes for knowledge base completion benefit from models that account for
  statistical uncertainty. We propose a probabilistic model for low rank tensor factorisation,
  which generalises the RESCAL model to explicitly model distributions of relations. This
  enables us to extend the particle Thompson sampling approach to knowledge bases, and we
  demonstrate the benefit of active knowledge acquisition. We explore additive and multiplicative
  approaches for composing relations, and show the regimes where compositional models are
  beneficial for knowledge base completion.
\end{abstract}

%!TEX root = ./cikm2016.tex

\section{Introduction}
\label{sec:intro}

%KG background from text, completion
Relational knowledge bases support reasoning, information retrieval, 
or question-answering tasks about entities and their relations.
Most of them contain facts in the form of (entity1, relation, entity2) triples, 
such as (CarlFriedrichGauss, BornIn, Braunschweig).
%(ThomasBayes, London, BornIn).
Automatically acquiring, maintaining, and reasoning in
knowledge bases is a very active topic area with many important and challenging research questions. 
%, has sustained attention from both academia and industry. 
Even the largest of such databases are known to be incomplete~\cite{dong2014knowledge}. %min2013distant
There are two main ways to fill in the missing facts:
the first learning new relations from large collections of text or hyper-text,
%~\cite{Mintz2009,carlson2010toward}, 
known as knowledge extraction, 
the second is inferring facts from existing relations, %~\cite{Lao2010,nickel2015review} 
known as knowledge completion. 

%There are two open 
One challenge in knowledge base completion is 
its disconnection from the knowledge extraction setting. 
It would be nice, for example, to know which question to query for in 
a search-based method for gathering triples~\cite{west2014knowledge}. 
Recently \cite{kajino2015active} propose an active learning strategy for completing 
knowledge triples; however the algorithm had problems simultaneously achieving high recall and faithful reconstruction.
% finds it difficult to achieve 
%high recall and high reconstruction at the same time. 
Having an active exploit-explore strategy would more effectively 
connect knowledge completion and extraction problems. 
Another challenge is leveraging compositional knowledge from existing triples. 
Also known as paths in knowledge graphs, composition of facts is key to 
achieving common reasoning tasks. 
For example, the two triples (CarlFriedrichGauss, BornIn, Braunschweig)
and (Braunschweig, LocatedIn, Germany) implies (CarlFriedrichGauss, BornIn Germany). 
Path ranking~\cite{Lao2010}, vector space traversal~\cite{guu2015traversing}
and composition~\cite{Neelakantan2015} techniques 
are recently developed to leverage such information for knowledge completion. 
We note, however, that a principled formulation that can address 
both open challenges is still missing -- namely, a relational model 
that can model knowledge compositions in an active setting. 

%\TODO{TODO}{revise summary-of-our-approach to be consistent w. the above}
We propose a novel probabilistic re-formulation of tensor factorisation,  
one of the main competitive variants for knowledge completion~\cite{nickel2015review}. 
%There are many advantages to a probabilistic formulation of tensor factorisation, 
%such as the quantification of uncertainty by the predictive distribution, 
%the ability to utilise priors, and the availability of principled model selection. 
We name our model probabilistic RESCAL (PRESCAL).
The probabilistic model provides a natural way of 
embracing uncertainty of triples that is crucial to develop 
%implementing
an active triple selection for knowledge completion, using  
Thompson sampling~\cite{scott10bandit} -- an approach for solving the multi-armed bandit problem,
which allows us to trade-off exploration and exploitation when identifying new triples.
%randomized probability matching, also known as Thompson sampling~\cite{scott10bandit}.
We also perform compositional training for the probabilistic model, 
by modeling relation compositions as algebraic operations in the probabilistic embedding space. 
For inference, we design Gibbs sampling for PRESCAL with and without compositions. 
%with by computing conditional posteriors. 
We employ a sequential Monte-Carlo method for the active querying of new triples, 
called particle Thompson sampling. 

We first test the proposed models with synthetic datasets. 
We observe that Thompson sampling provides significant gain over random sampling of triples; 
we also observe a clear gain in tensors with known composition structure. 
We then evaluated the model on three real-world relation datasets. In the passive learning setting, 
we find PRESCAL outperforming the non-probabilistic version of tensor factorisation, 
and that compositions help when the training set is sparse. 
In the active learning scenario, 
we find that PRESCAL achieves the highest cumulative gain across all datasets. 
It is encouraging to see the exploitation-exploration strategy with uncertainty outperforming 
active learning strategies that focus solely on exploitation or exploration. 

We are pleased to be able to learn a vector-space model for entities and relations with uncertainty, 
bridging the gap of probabilistic active sampling and knowledge compositions. 
We look forward to follow-on work with better sampling strategies and computational scalability. 

\eat{
Statistical relational models have been proposed to tackle
various problems of knowledge bases. For example, it can be used 
for question understanding and answering problems ,
or it can be integrated into a search engine
to improve the users' search experiment \cite{dong2014knowledge}.

However, most of previous latent feature models lack the consideration
of constructing a knowledge graph with the latent features.
As an alternative approach, distant supervision algorithms
\cite{Mintz2009} suggest a way to fill a missing part of knowledge graph
through the extensive NLP processing, however, this method inherently
requires a set of initial observations used as a distant supervision.

Here, we tackle the problem of knowledge base construction with the
statistical relational model.
A goal of knowledge base construction is to acquire a maximum number
of valid triples with a limited budget.
This goal corresponds to a general objective of the multi-armed bandit (MAP)
problem where we want to minimise the cumulative regret or maximise
the cumulative reward over time. 
In recent years, Thompson sampling has been emerged as  
an competitive solution in MAP problems.
We borrow Thompson sampling, which provides a principled
way to find an optimal trade off between exploration and exploitation,
to solve the knowledge base construction problem.
}

\eat{
We propose a compositional relation model that exploit the compositional structure of 
knowledge graph to capture the latent semantic structure of the entities and relations.
While previously suggested vector space models provide a statistical way to infer the latent semantic 
structure of entities and relations, but lack consideration of a graph structure of a knowledge base itself.

In a separate way from the vector space models, graph feature algorithms such as the path ranking algorithm 
are suggested to fill a missing part of a knowledge graph \cite{Lao2010}. The graph feature algorithms 
directly include graph structures, such as edge type, node type, and node degree, to learn and predict new 
triples, however, the absence of latent structure makes the models failed to predict a new triple when the 
target entities does not have rich structural background\cite{nickel2015review}.

We propose a compositional vector space model that benefits the latent representation of vector space model 
along with the graph structure of the graph feature models. Recently, Guu et. al. suggest a compositional 
training framework for vector space models \cite{gu2015traversing}, where paths over a knowledge graph act 
as a new form of structural regularisation of the models. Based on their work, we extend the compositional 
approach within a probabilistic framework with two compositional structures.
}

\eat{%% v1 of intro
As the amount of information codified in a computer readable fashion increases, the management
of knowledge bases need to become increasingly more automated. In this paper, we study
the problem of acquiring new knowledge given an existing knowledge base. Knowledge bases
are modeled as a set of relations between pairs of entities, for example the factoid
``Barack Obama is the 44th president of the United States''
is modeled as two entities (Barack Obama, United States) being related by ``president of''.
Such relations have a natural representation as a sparse graph or a tensor of order 3.
Given this representation, we model the acquisition of new knowledge as the
identification of new triples that capture particular relations between two entities.

There are several challenges when we apply machine learning methods to completing
existing knowledge bases, namely:
sparse, noisy, and incomplete annotations.
There has been recent success in transferring ideas from matrix completion problems to
the tensor domain to overcome the challenge of sparsity~\cite{unknown}.
We follow this thread of research by using a low rank approximation model for tensor
factorisation. Such low rank approximations can also be seen as a latent variable probabilistic
model, which additionally captures the inherent uncertainty of noisy annotations.
We propose a probabilistic model for tensor factorisation and explore both the Gaussian
and Logistic model. The probabilistic model provides a natural way of implementing
randomized probability matching, also known as Thompson sampling~\cite{scott10bandit}.
Thompson sampling is an approach for solving the multi-armed bandit problem,
which allows us to trade off exploration and exploitation when identifying new triples.
This provides a principled approach to identify promising candidates for knowledge base
completion.
We additionally consider compositional relations as an additional source of weak information
to further utilise the existing (incomplete) knowledge items.

Goal of this paper 1: Populating knowledge graph with an active label acquisition process 
corresponding to [B,C,A] in Figure \ref{fig:related3d}.

[B,C,P] could be an alternative direction (or both).
}

%!TEX root = ./cikm2016.tex

%\section{Related Work}
\textbf{Related work}:
The literature on data factorisation and vector space models for 
relational data is vast. 
%or low-rank approximation is vast. 
We give a brief overview of related work along three design choices:
%with respect to three orthogonal dimensions, 
the method, the learning strategy, and the data representation. 
We then use these dimensions to help position our work. 
\begin{description}
%\item[Statistical method] With an statistical assumption used to develop 
%underlying model structure, we broadly categorise models into Bayesian 
%and non-Bayesian models.
\item[Probabilistic/Non-Probabilistic] This refers to two broad classes of model 
formulation, whether an obtained model has a probabilistic interpretation. 
%The probabilistic models are capable of placing priors and measuring uncertainty.
\item[Passive/Active] This refers to two different learning strategies, 
of passively learning a model given labeled data points, or actively 
requesting data points to be labelled.
\item[Matrix/Tensor/Composition] Relational learning problems can operate on 
different data representations. Matrix representation is common when a dataset 
can be represented as a bi-partite graph, such as (user, item). 
Tensor representation is handy when edges in the 
graph have labels, i.e. (entity1, relation, entity2). We can think of 
compositions as paths in the graph, i.e. entity1 -- relation1 -- entity2 -- 
relation2 -- entity3.
%\item[Data representation] Data for a factorisation problem can be 
%represented as a matrix or tensor. A graph representation of tensor reveals
%a compositional structure of the data.
\end{description}
In Table \ref{tbl:relatedwork}, we summarise a sample of related recent work 
along all combinations in each dimension. 
%Note that N-Pr, A, C, or Non-Probabilistic Active Composition model, can be done by 
%simply using the query strategies from \cite{kajino2015active} on the 
%compositional model by \cite{guu2015traversing}. 
Our work address a critical gap 
in probabilistic Tensor factorisation capable of learning in the 
Active setting with relation Compositions.
\eat{The knowledge graph construction has been widely focused in the natural 
language processing communities. 
Manual data acquisition with human crafted 
or automatically inferred synthetic rules or distant supervision on a large
amount of unlabelled text are the main tools to construct \cite{fader2011identifying,Mintz2009}.}

\eat{
Given this position, our work is inspired by, and most closely related to:
active multi-relational data construction (AMDC) 
with tensor factorisation~\cite{kajino2015active}, 
Thomson sampling for matrix factorisation~\cite{kawale2015efficient}, 
and compositional objectives in vector space~\cite{guu2015traversing}.
Note that we reformulate the compositional objectives probabilistically such that it can be used in an active setting; 
our approach for active learning is a generalisation of
Thomson sampling from matrixes to tensors; 
AMDC find that reconstruction accuracy and knowledge population 
cannot be achieved at the same time with strategies geared towards 
either exploit or reducing uncertainty, 
we show that the two objectives can be achieved at the same time with 
a properly designed exploration-exploitation scheme. 
}
\eat{show active knowledge graph construction
algorithm based on the low-rank structure. They proposed active 
multi-relational data construction (AMDC) with two separate problems: 
a knowledge base population and predictive model construction, and show 
that these two goals cannot be achieved at the same time. In this work, 
we show that it can be achieved at the same time with a properly designed
exploration and exploitation scheme that has been shown in the matrix 
factorisation problem \cite{kawale2015efficient}.
}


%\begin{itemize}
%\item [B,A,M] Efficient Thompson Sampling for OnlineMatrix-Factorization Recommendation\cite{kawale2015efficient}. Active learning and search on low-rank matrices \cite{sutherland2013active}. Collaborative filtering as multi armed bandit \cite{guillou2015collaborative}
%\item [N,A,M] Matrix completion with queries \cite{ruchansky2015matrix}.
%\item [B,P,M] PMF \cite{mnih2007probabilistic} ...
%\item [N,P,M] NMF\cite{lee1999learning} ...
%\item [B,P,T] Bayesian Tensor Factorisation models. CANDECOMP/PARAFAC (CP) decomposition \cite{xiong2010temporal,schmidt2009probabilistic}, CP and TUCKER3 \cite{yilmaz2012algorithms}
%\item [N,A,T] Populating knowledge graph with active learning (IBM) \cite{kajino2015active}
%\item [N,P,T] Rescal \cite{nickel2011three}, TransE \cite{bordes2013translating}, and many others.
%\item [N,P,C] Compositional vector space model \cite{Neelakantan2015}, 
%\end{itemize}
%
%Might be relevant, but not positioned in the figure.
%\begin{itemize}
%\item Clustering based Bayesian approach for learning relations: Infinite relational model based on entity clustering \cite{kemp2006learning}.
%\item Path ranking algorithm (graph feature model) \cite{Lao2010}
%\end{itemize}

%\begin{figure}[t]
%	\centering
%	\includegraphics[width=\linewidth]{images/3d_plot.pdf}			
%	\caption{\label{fig:related3d}Scope of our work}
%\end{figure}

%!TEX root = ./cikm2016.tex

\begin{table}[t]
\centering
\caption{\label{tbl:relatedwork}The categorisation of factorisation problems with respect to 
three design considerations. The column headings are Probabilistic(Pr)/Non-Probabilistic(N-Pr) method, Passive(P)/Active(A) learning, and Matrix(M)/Tensor(T)/Compositional(C) structure. In this work, we tackle the problems denoted by an asterisk.}
\vskip 0.15in
\begin{tabular}{c c c l}
Pr/N-Pr & P/A & M/T/C & References	\\ \hline \hline

N-Pr & P & M & \cite{lee1999learning}\\ \hline
N-Pr & A & M & \cite{ruchansky2015matrix}\\  \hline

N-Pr & P & T& \cite{nickel2011three}\cite{kolda2009tensor}\\ \hline
N-Pr & A & T & \cite{kajino2015active} \\  \hline
N-Pr & P & C & \cite{Neelakantan2015}\cite{guu2015traversing}\\ \hline
N-Pr & A & C & -- \\ \hline
Pr & P & M & \cite{mnih2007probabilistic}\\ \hline
Pr & A & M&  \cite{kawale2015efficient}\cite{sutherland2013active}\\ \hline

Pr & P & T& *, \cite{xiong2010temporal}\cite{schmidt2009probabilistic} \\ \hline
Pr & A & T & * \\ \hline
Pr & P & C & * \\ \hline
Pr & A & C & *
\end{tabular}
\end{table}
%!TEX root = ./cikm2016.tex
\section{Probabilistic RESCAL}
\label{sec:brescal}

\begin{table*}[bt]
\caption{Parameters for Gibbs updates. The conditional of $e_i$ and $R_k$ follows the normal distribution with mean $\mu$ and precision matrix $\Lambda$. $\otimes$ is the Kronecker product.}
\label{tab:brescalposterior}
\vskip 0.05in
\begin{tabu}{l|l|l|l}
var&$\mu$&$\Lambda$&$\xi$\\
\hline
$e_i$&
$\frac{1}{\sigma_x^2}\Lambda_i^{-1}\xi_i$&
$\frac{1}{\sigma_x^2} \sum_{jk : x_{ikj} \in \mathcal{X}^{t}} (R_k e_j)(R_k e_j)^\top$&
$\sum_{jk : x_{ikj} \in \mathcal{X}^{t}}  x_{ikj} R_{k} e_{j} +
\sum_{jk : x_{jki} \in \mathcal{X}^{t}} x_{jki} R_{k}^\top e_{j}.$
\\
$vec(R_k)$&
$\frac{1}{\sigma_x^2}\Lambda_k^{-1}\xi_k$&
$\frac{1}{\sigma_x^2} \sum_{ij:x_{ikj} \in \mathcal{X}^{t}} (e_i
\otimes e_j)(e_i \otimes e_j)^\top + \frac{1}{\sigma_r^2} {I}_{D^2}$&
$\sum_{ij:x_{ikj} \in \mathcal{X}^{t}} x_{ikj} (e_{i} \otimes e_{j}).$
\end{tabu}
\end{table*}


A relational knowledge base consists of a set triples in the form of $(i, k, j)$
where $i$, $j$ are entities, and $k$ is a relation. A triple can be distinguished
in a valid triple and invalid triple based on a semantic meaning of the triple. An
example of the valid triple in Freebase is \texttt{(BarackObama, PresidentOf, U.S.)}, and an
example of the invalid triple is \texttt{(BarackObama, PresidentOf, U.K.)}.
A knowledge base can be represented in a three-way binary tensor
$\mathcal{X} \in \{0, 1\}^{N \times K \times N}$, where $K$ is a number of
relations, $N$ is a number of entities, and $x_{ikj}\in \{0, 1\}$ indicates whether
the triple is valid.

We model the entity $i$ as vectors $e_i$ and the relation $k$ as matrix $R_k$ with an
appropriately chosen latent dimension $D$. This follows a popular model
for statistical relational learning, which is to factorise the tensor into a
set of latent vector representations, such as the bilinear model RESCAL~\cite{nickel2011three}.
RESCAL aims to factorise each relational slice $X_{:k:}$ into a set of rank-$D$ latent
features as follows:
\[
  \mathcal{X}_{:k:} \approx E R_k E^\top, \qquad \text{for } k = 1, \dots, K
\]
Here, $E\in {\mathbb R}^{N \times D}$ contains the latent features of the
entities $e_1, \ldots, e_N$ and $R_k\in {\mathbb R}^{D \times D}$ models the interaction of the
latent features between entities in relation $k$.

We propose a probabilistic framework that directly generalises RESCAL
by placing priors over the
latent features. For each entity $i$, the latent feature of an entity $e_i \in
\mathbb{R}^{D}$ is drawn from an isotropic multivariate-normal distribution.
\begin{align}
\label{eqn:entity_gen}
e_i \sim {N}(\mathbf{0}, \sigma_e^2{I}_D)
\end{align}
For each relation $k$, we draw matrix $R_k$ from
a zero-mean isotropic matrix normal distribution.
\begin{align}
\label{eqn:relation_gen}
R_k \sim \mathcal{MN}_{D \times D}(\mathbf{0}, \sigma_r{I}_D, \sigma_r{I}_D) \\
\text{or equivalently}\enspace r_k  = \text{vec}(R_k) \sim N(\mathbf{0}, \sigma_r^2 I_{D^2}) \notag
\end{align}
where $\text{vec}(R_k)$ denotes the flattening of the matrix.

We consider two observation models for $x_{ikj}$: real or binary variables. By placing a
normal distribution over $x_{ikj}$,
\begin{align}
  x_{ikj} |e_i, e_j, R_k \sim \mathcal{N}(e_i^{\top} R_k e_j, \sigma_x^2),\label{eqn:triple_gen}
\end{align}
we model the value of triple as a real variable.
This is not a natural choice since the triple is a binary variable, however, we can control the confidence on different observations
through the variance parameter $\sigma_x^2$.
%The role of this parameter will be further discussed in the compositional model section.

We develop a Gibbs sampler to perform the posterior inference for the probabilistic RESCAL (PRESCAL). The conditional distribution of each latent variable is given by:
\begin{align}
p(e_i |E_{-i}, \mathcal{R}, \mathcal{X}^{t}, \sigma_e, \sigma_x) &= \mathcal{N}(e_i | \mu_i,
\Lambda_i^{-1})  \label{eqn:sample_e} \\
p(R_k|E, \mathcal{X}, \sigma_r, \sigma_x) &= \mathcal{N}(\text{vec}(R_k) |
\mu_k, \Lambda_k^{-1}) \label{eqn:sample_r}
\end{align}
where the negative subscript $-i$ indicates the every other entity variables except entity $i$.
Exact forms of the posterior means and precision matrices are listed in Table~\ref{tab:brescalposterior}, where we have
used the identity $e_i^{\top} R_k e_j = r_k^{\top} e_i \otimes e_j$.

Alternatively, we may want to model the binary observation more precisely.
Therefore we model $x_{ikj}$ as a binomial random variable whose
probability is determined by logistic regression:
\[
p(x_{ikj}=1) = \sigma(e_i^{\top} R_k e_j),
\]
where $\sigma$ is a sigmoid function.
We approximate the conditional posterior of
$E$ and $R$ by Laplace approximation~\cite{bishop2006pattern}. The maximum a
posterior estimate of $e_i$ or $R_k$ given the rest can be computed through 
standard logistic regression solvers with the priors over $e_i$ and $R_k$ as regularisation parameters. Given the
maximum a posteriori parameters $e_i^*$, the posterior covariance $S_i$ of entity
$i$ takes the form
\begin{align*}
S_i^{-1} = \sum_{x_{ikj}} \sigma(e_{i}^{*\top} R_k e_{j}) (1 - \sigma(e_{i}^{*\top} R_k e_{j})) R_k
e_{j}(R_k e_{j})^\top\\
 + \sum_{x_{jki}} \sigma(e_{j}^{\top} R_k e_{i}^*) ( 1- \sigma(e_{j}^{\top} R_k e_{i}^*)) R_k^\top e^*_{i}(R_k^\top e^*_{i})^\top + I\sigma_e^{-1}
.
\end{align*}
The posterior covariance of $R_k$ can be computed in the same way. Let $R_k^*$ is a maximum a posterior solution of $R_k$ given $E$. Then, the conditional posterior covariance $S_k$ of relation $k$ has the form of:
\begin{align}
S_k^{-1} = \sum_{x_{ikj}} \sigma(e_{i}^{\top} R_k^* e_{j}) (1 - \sigma(e_{i}^{\top} R_k^* e_{j})) 
\bar{e}_{ij}\bar{e}_{ij}^\top + I\sigma_r^{-1}, \notag
\end{align}
where $\bar{e}_{ij} = e_i \otimes e_j$.


The probabilistic view of tensor factorisation has many advantages such as
the quantification of uncertainty by the predictive distribution,
the ability to utilise priors, and
the availability of principled model selection.
We show in the empirical experiments that PRESCAL outperforms standard RESCAL.

%!TEX root = ./icml2016.tex

\section{Compositional Relations}

\begin{figure}[t]
	\centering
	\includegraphics[width=\linewidth]{images/comp_training_error_kinship_small.pdf}
	\includegraphics[width=\linewidth]{images/comp_training_error_umls_small.pdf}			
	\includegraphics[width=\linewidth]{images/comp_training_error_nation_small.pdf}				
	\caption{\label{fig:r_vs_br} ROC-AUC scores of compositional models. The x-axis denotes the proportion of an observed triples including negative triples used for training models. We  use another 30\% of triples to evaluate the ROC-AUC score. In general, compositional models outperform BRESCAL and BLOGIT model when the size of training set is relatively small, whereas BRESCAL and BLOGIT perform slightly better than or comparable to the compositional models when the size of training set is relatively large. For UMLS dataset, the multiplicative compositional model consistently outperforms the other models across all training proportions.}
\end{figure} 


From two triples:
$(i, k_1 ,j)$,  $(j, k_2, l)$

Compositional triple:
$(i, {c}(k_1, k_2), l)$

This compositional triple may have multiple path over the knowledge graph. For example, both pairs of triples $(i, k_1 ,j)$,  $(j, k_2, l)$ and $(i, k_1 ,m)$,  $(m, k_2, l)$ can be represented as $(i, {c}(k_1, k_2), l)$. To include the compositionality of the knowledge graph, we set  $x_{(i, {{c}(k_1, k_2)}, l)}$ be the number of path from entity $i$ to entity $j$ through relation $k_1$ and $k_2$.

We model a degree of the compositional relation as
\begin{align}
x_{(i, {{c}(k_1, k_2)}, l)} \sim \mathcal{N}(e_i^\top R_{{c}(k_1,k_2)} e_j, \sigma_{c}^2),
\end{align}
where $R_{{c}(k_1,k_2)} \in \mathbb{R}^{D\times D}$.

Let $\mathcal{C}^{L}$ be a set of all possible compositions of which length is up to length $L$, $c \in \mathcal{C}$ be a sequence of compositions, and $c(i)$ be $i$th index of a relation in sequence $c$. With the set of compositions $\mathcal{C}^{L}$, we can expand set of observed triples $\mathcal{X}^{t}$ to set of observed compositional triple $\mathcal{X}^{\mathcal{C}^{L}(t)}$ of which element $x_{icj}$ is a number of path from entity $i$ to entity $j$ through sequence of relations $c$ in $\mathcal{X}^{t}$.

\subsection{Additive Compositionality}
Let $R_{{c}} = \frac{1}{|c|}(R_{c(1)} + R_{c(2)} + \dots + R_{c(|c|)})$, then
\begin{align}
x_{(i, k_{{c}}, l)} \sim \mathcal{N}(e_i^\top R_c e_j, \sigma_{c}^2).
\end{align}

The conditional distribution of $e_i$ given $E_{-i}, \mathcal{R}, \mathcal{X}^{t}, \mathcal{X}^{L(t)}$ is 
\begin{align} \label{eqn:sample_e}
p(e_i |E_{-i}, \mathcal{R}, \mathcal{X}^{t}, \mathcal{X}^{L(t)}) &= \mathcal{N}(e_i | \mu_i, \Lambda_i^{-1}),
\end{align}
where
\begin{align*}
\mu_i &= \Lambda_i^{-1}\xi_i \\
\Lambda_i &= \frac{1}{\sigma_x^2} \sum_{jk : x_{ikj} \in \mathcal{X}^{t}} (R_k e_j)(R_k e_j)^\top \\
&\quad+ \frac{1}{\sigma_x^2} \sum_{jk : x_{jki} \in \mathcal{X}^{t}} (R_k^\top e_j)(R_k^\top e_j)^\top \\
&\quad + \frac{1}{\sigma_c^2} \sum_{jc : x_{icj} \in \mathcal{X}^{L(t)}} (R_c e_j)(R_c e_j)^\top \\
&\quad+ \frac{1}{\sigma_c^2} \sum_{jc : x_{jci} \in \mathcal{X}^{L(t)}} (R_c^\top e_j)(R_c^\top e_j)^\top + \frac{1}{\sigma_e^2} {I}_D \\
\xi_i &= \frac{1}{\sigma_x^2}\sum_{jk : x_{ikj} \in \mathcal{X}^{t}}  x_{ikj} R_{k} e_{j} + \frac{1}{\sigma_x^2}\sum_{jk : x_{jki} \in \mathcal{X}^{t}} x_{jki} R_{k}^\top e_{j} \\
& + \frac{1}{\sigma_c^2}\sum_{jc : x_{icj} \in \mathcal{X}^{L(t)}}  x_{icj} R_{c} e_{j} + \frac{1}{\sigma_c^2}\sum_{jc : x_{jci} \in \mathcal{X}^{L(t)}} x_{jci} R_{c}^\top e_{j}
\end{align*}

To compute the conditional distribution of $R_k$, we first decompose $R_c$ into two part where $R_c = \frac{1}{|c|} R_k + \frac{|c|-1}{|c|}R_{c/k}$, where $R_{c/k} = \sum_{k' \in c/k} R_{k'}$, if compositional sequence $c$ contains relation $k$. Vectorisation of $R_c$ and $R_{c/k}$ are represented as $r_c$ and $r_{c/k}$, respectively.

The degree of compositional path is represented with a decomposition as follows:
\begin{align}
x_{(i, c, l)} \sim \mathcal{N}(e_i^\top (\frac{1}{|c|} R_k + \frac{|c|-1}{|c|}R_{c/k}) e_j, \sigma_{c}^2).
\end{align}

Then, the conditional distribution $R_k$ given $R_{-k}, E, \mathcal{X}^{t}, \mathcal{X}^{L(t)}$ is
\begin{align}
\label{eqn:comp_cond_r}
p(R_k|E, \mathcal{X}^{t}, \mathcal{X}^{L(t)}, \sigma_r, \sigma_x)  &= \mathcal{N}(\text{vec}(R_k) | \mu_k, \Lambda_k^{-1}),
\end{align}
where
\begin{align*}
\mu_k &=\Lambda_k^{-1}\xi_k \\
\Lambda_k &= \frac{1}{\sigma_x^2} \sum_{ij:x_{ikj} \in \mathcal{X}^{t}} \bar{e}_{ij}\bar{e}_{ij}^\top + \frac{1}{\sigma_r^2} {I}_{D^2} \\
& +\frac{1}{|c|^2 \sigma_c^2} \sum_{ij:x_{icj} \in \mathcal{X}^{L(t)},\text{ }k \in c} \bar{e}_{ij} \bar{e}_{ij}^\top \\
\xi_k &=  \frac{1}{\sigma_x^2}\sum_{ij:x_{ikj} \in \mathcal{X}^{t}} x_{ikj} \bar{e}_{ij}\\
& +\frac{1}{|c| \sigma_c^2} \sum_{ij:x_{icj} \in \mathcal{X}^{L(t)},\text{ }k \in c} x_{icj} \bar{e}_{ij} - \frac{|c|-1}{|c|} \bar{e}_{ij} r_{c/k}^\top \bar{e}_{ij}\\
\bar{e}_{ij} &= e_{i} \otimes e_{j}.
\end{align*}

Note that the ordering of the relations in compositional sequence $c$ does not affect the value of compositional triple $(i, c, j)$.

\subsection{Multiplicative Compositionality}
Let compositional relation $R_c = R_{c(1)} R_{c(2)} \dots R_{c(|c|)}$.

If the determinant of latent relation $R_k$ is greater than 1, the compositional latent relation $R_c$ might be exploded after multiplying a long sequence of relations. To obtain a stable scale of compositional relation $R_c$, one may multiply decaying factor $\tau < 1$ after each composition. $R_c = \tau^{|c|-1} R_{c(1)} R_{c(2)} \dots R_{c(|c|)}$.

\begin{align}
x_{(i, c, j)} \sim \mathcal{N}(e_i^\top R_{c(1)}R_{c(2)} \dots R_{c(|c|-1)}R_{c(|c|)} e_j, \sigma_{c}^2)
\end{align}

If the compositional sequence $c$ contains relation $k$, the mean parameter of normal distribution contains $R_k$ in the middle of compositional sequence (i.e., $e_i^\top R_{c(1)}R_{c(2)} \dots R_{c(\delta_k)} \dots R_{c(|c|-1)}R_{c(|c|)} e_j$ where $\delta_k$ is the index of relation $k$). For notational simplicity, we will denote the left side $e_i^\top R_{c(1)}R_{c(2)} \dots R_{c(\delta_k -1)}$ as $\bar{e}_{ic(:\delta_k)}^\top$, and the right side $R_{c(\delta_k + 1)} \dots R_{c(|c|-1)}R_{c(|c|)} e_j$ as $\bar{e}_{ic(\delta_k:)}$, therefore we can rewrite the mean parameter as $\bar{e}_{ic(:\delta_k)}^\top R_{k} \bar{e}_{ic(\delta_k:)}$.

The conditional distribution of $e_i$ given the rest is the same as Equation $\ref{eqn:sample_e}$. The conditional of $R_k$ is
\begin{align}
p(R_k|E, \mathcal{X}, \sigma_r, \sigma_x)  &= \mathcal{N}(\text{vec}(R_k) | \mu_k, \Lambda_k^{-1}),
\end{align}
where
\begin{align*}
\mu_k &= \Lambda_k^{-1}\xi_k \\
\Lambda_k &= \frac{1}{\sigma_x^2} \sum_{ij:x_{ikj} \in \mathcal{X}^{t}} (e_i \otimes e_j)(e_i \otimes e_j)^\top + \frac{1}{\sigma_r^2} {I}_{D^2} \\
+ &\frac{1}{\sigma_c^2} \sum_{ij:x_{icj} \in \mathcal{X}^{L(t)}, \text{ }k \in c} (\bar{e}_{ic(:\delta_k)} \otimes \bar{e}_{jc(\delta_k:)})(\bar{e}_{ic(:\delta_k)} \otimes \bar{e}_{jc(\delta_k:)} )^\top \\
\xi_k &= \frac{1}{\sigma_x^2} \sum_{ij:x_{ikj} \in \mathcal{X}^{t}} x_{ikj} (e_{j} \otimes e_{i}) \\
& + \frac{1}{\sigma_c^2} \sum_{ij:x_{icj} \in \mathcal{X}^{L(t)}, \text{ }k\in c} x_{icj} (\bar{e}_{ic(:\delta_k)}  \otimes \bar{e}_{jc(\delta_k:)}).
\end{align*}


Thoughts: As the length of sequence $c$ increases, a small error in the first few multiplication will result a large differences in the final compositional relation. One way to mitigate this cascading error is to increase the variance of compositional triples as length of sequences increase.


%!TEX root = ./icml2016.tex
\section{Particle Thompson Sampling}
Let $\mathcal{X}^{t}$ be a set of observed triples up to time $t$.

The conditional distribution of $e_i$ given $\mathcal{R}$ and other entities $E_{-i}$
\begin{align} \label{eqn:sample_e}
p(e_i |E_{-i}, \mathcal{R}, \mathcal{X}^{t}, \sigma_e, \sigma_x) &= \mathcal{N}(e_i | \mu_i, \Lambda_i^{-1}),
\end{align}
where
\begin{align*}
\mu_i &= \frac{1}{\sigma_x^2}\Lambda_i^{-1}\xi_i \\
\Lambda_i &= \frac{1}{\sigma_x^2} \sum_{jk : x_{ikj} \in \mathcal{X}^{t}} (R_k e_j)(R_k e_j)^\top \\
&\quad+ \frac{1}{\sigma_x^2} \sum_{jk : x_{jki} \in \mathcal{X}^{t}} (R_k^\top e_j)(R_k^\top e_j)^\top+ \frac{1}{\sigma_e^2} {I}_D \\
\xi_i &= \sum_{jk : x_{ikj} \in \mathcal{X}^{t}}  x_{ikj} R_{k} e_{j} + \sum_{jk : x_{jki} \in \mathcal{X}^{t}} x_{jki} R_{k}^\top e_{j}.
\end{align*}
The conditional distribution of $R_k$ given $E$
\begin{align}
\label{eqn:sample_r}
p(R_k|E, \mathcal{X}, \sigma_r, \sigma_x)  &= \mathcal{N}(\text{vec}(R_k) | \mu_k, \Lambda_k^{-1}),
\end{align}
where
\begin{align*}
\mu_k &= \frac{1}{\sigma_x^2}\Lambda_k^{-1}\xi_k \\
\Lambda_k &= \frac{1}{\sigma_x^2} \sum_{ij:x_{ikj} \in \mathcal{X}^{t}} (e_i \otimes e_j)(e_i \otimes e_j)^\top + \frac{1}{\sigma_r^2} {I}_{D^2} \\
\xi_k &= \sum_{ij:x_{ikj} \in \mathcal{X}^{t}} x_{ikj} (e_{i} \otimes e_{j}).
\end{align*}

The posterior marginal predictive distribution of $x_{ikj}$ given $\mathcal{X}$ and $E$.
\begin{align}
\label{eqn:marginal_predict}
&p(x_{ikj}| E, \mathcal{X}^{t}, \sigma_x, \sigma_r) \\
&= \mathcal{N}(x_{ikj}| \mu_k ^\top (e_i \otimes e_j), \frac{1}{\sigma_x^2} +  (e_i \otimes e_j)^\top \Lambda_k (e_i \otimes e_j)) \notag
\end{align}

\subsection{Particle Thompson sampling with MCMC kernel}
Let $H$ be the number of particles, and $\Theta= \{E, \mathcal{R}\}$ be a set of latent features.
%Algorithm \ref{alg:smc} describes basic particle Thompson sampling for the tensor factorisation.
Algorithm \ref{alg:rbsmc} describes Rao-Blackwellized particle Thompson sampling where relation matrix $R_k$ is marginalized out.

Under the mild assumption where $p(\Theta | \mathcal{X}^{t-1}) \approx p(\Theta | \mathcal{X}^{t})$, the weight of each particle at time $t$ can be computed as follows \cite{del2006sequential,chopin2002sequential}:
\begin{align}
w_{h}^{t} = \frac{p(\mathcal{X}^{t} | \Theta)}{p(\mathcal{X}^{t-1} | \Theta)} = p(x^{t} | \Theta, \mathcal{X}^{t-1})
\end{align}


%\subsection{(?)Particle Thompson sampling with SGLD kernel}
%Above two algorithms are not well suitable for large scale dataset because the sample requires quantities estimated over all possible triples.
%
%Is it possible to use the stochastic gradient Langevin dynamics (SGLD) \cite{welling2011bayesian} kernel $K(E' | E)$ to sample $E'$ given $E$ with or without auxiliary variable $R_k$?
%\begin{align}
%e_i' \leftarrow e_i + \frac{\epsilon_t}{2}\Bigg\{\nabla \log p(\mathcal{X}|e_i) + \nabla \log p(e_i|\sigma_e)\Bigg\} + \nu_t
%\end{align}
%where, $\nu_t \sim \mathcal{N}(0, \epsilon_t I)$.

%\begin{algorithm}[t!]
%   \caption{Particle Thompson Sampling for Tensor Factorisation}
%   \label{alg:smc}
%\begin{algorithmic}
%   \STATE {\bfseries Input:} $\mathcal{X}, \sigma_x, \sigma_e, \sigma_r$.
%   \FOR{$t=1,2, \dots$}
%   \STATE \textit{Thompson Sampling}:
%   \STATE $h_t \sim $ Cat$(\mathbf{w}^{t})$
%   \STATE $(i,k,j) \leftarrow \arg\max p(x_{ikj}| E^{h_t}, \mathcal{R}^{h_t})$
%   \STATE Observe $x_{ikj}$ and update $\mathcal{X}$
%
%   \STATE \textit{Particle Filtering}:
%   \STATE $\forall h, w_h^{t+1} \propto p(x_{ikj} | E^{h}, \mathcal{R}^{h})$   \hfill $\triangleright$ Reweighting
%   \IF{ESS$(\mathbf{w}^{t+1}) \leq N$}
%   \STATE resample particles
%   \STATE $w_h^{t+1} \leftarrow 1/H$
%   \ENDIF
%
%   \FOR{$h=1$ {\bfseries to} $H$}
%   \STATE $\forall k, R_k^{h} \sim p(R_k | \mathcal{X}, E^{h})$   \hfill $\triangleright$ see Eq. (\ref{eqn:sample_r})
%   \STATE $\forall i, e^{h}_i \sim p(e_i | \mathcal{X}, E^{h}_{-i}, \mathcal{R}^{h})$ \hfill $\triangleright$ see Eq. (\ref{eqn:sample_e})
%   \ENDFOR
%
%   \ENDFOR
%\end{algorithmic}
%\end{algorithm}

\begin{algorithm}[t!]
   \caption{Rao-Blackwallized Particle Thompson Sampling for Tensor Factorisation}
   \label{alg:rbsmc}
\begin{algorithmic}
   \STATE {\bfseries Input:} $\sigma_x, \sigma_e, \sigma_r$.
   \FOR{$t=1,2, \dots$}
   \STATE \textit{Thompson Sampling}:
   \STATE $h_t \sim $ Cat$(\mathbf{w}^{t})$
   \STATE $(i,k,j) \leftarrow \arg\max p(x_{ikj}| E^{h_t})$    \hfill $\triangleright$ see Eq. (\ref{eqn:marginal_predict})
   \STATE Observe $x_{ikj}$ and update $\mathcal{X}^{t}$

   \STATE \textit{Particle Filtering}:
   \STATE $\forall h, w_h^{t+1} \propto p(x_{ikj} | E^{h})$   \hfill $\triangleright$ Reweighting, Eq. (\ref{eqn:marginal_predict})
   \IF{ESS$(\mathbf{w}^{t+1}) \leq N$}
   \STATE resample particles
   \STATE $w_h^{t+1} \leftarrow 1/H$
   \ENDIF

   \FOR{$h=1$ {\bfseries to} $H$}
   \STATE $\forall k, R_k^{h} \sim p(R_k | \mathcal{X}^{t}, E^{h})$   \hfill $\triangleright$ Auxiliary sampling, see Eq. (\ref{eqn:sample_r})
   \STATE $\forall i, e^{h}_i \sim p(e_i | \mathcal{X}^{t}, E^{h}_{-i}, \mathcal{R}^{h})$ \hfill $\triangleright$ see Eq. (\ref{eqn:sample_e})
   \ENDFOR

   \ENDFOR
\end{algorithmic}
\end{algorithm}


%!TEX root = ./icml2016.tex
\section{Experiments}
\subsection{Synthetic data}
\begin{figure}[t]
	\centering
	
	\subfigure[E=5, K=5, D=5\label{fig:syn1}]{
	\includegraphics[width=0.42\linewidth]{images/toy_5_5_5.pdf}
	}
	\subfigure[E=10, K=10, D=5\label{fig:syn2}]{
	\includegraphics[width=0.45\linewidth]{images/toy_10_10_5.pdf}				
	}
%	\includegraphics[width=0.32\linewidth]{images/toy_5_10_5.pdf}			
%	\includegraphics[width=0.32\linewidth]{images/toy_10_5_5.pdf}				
	\caption{\label{fig:synthetic} Cumulative regret of particle Thompson sampling on different sizes of the synthetic dataset. The synthetic dataset is generated by the model assumption (Eq. \ref{eqn:entity_gen} - \ref{eqn:triple_gen}). We compared the particle Thompson sampling with random sampling method. The averaged cumulative regrets are plotted with one standard error. As the model obtained more and more labeled samples from Thompson sampling, the cumulative regrets are converged. This result  indicates the particle sampling correctly inferred latent features of entities and relations on the synthetic datasets. // The results are averaged over 10 runs. $\sigma_e = 10$, $\sigma_r=10$, $\sigma_x=0.1$, $H=5$.}
\end{figure}


\subsection{particle Thompson sampling}
\begin{figure}[t]
	\centering
	
	\subfigure[KINSHIP]{
	\includegraphics[width=\linewidth]{images/thompson_kinship.pdf}
	}
	\subfigure[UMLS]{
	\includegraphics[width=\linewidth]{images/thompson_umls.pdf}				
	}
	\subfigure[NATION]{
	\includegraphics[width=\linewidth]{images/thompson_nation.pdf}				
	}	
	\caption{Placeholder for the future results. Will include the cumulative gain and ROC-AUC score of the developed models with the active and passive learning methods. We will see how the compositional model performs to compare with other models without any initial observation (May include the IBM model without any initial observation).}
\end{figure}

\subsection{Compositional ...}
\begin{figure}[t]
	\centering
	\includegraphics[width=\linewidth]{images/comp_training_error_kinship_small.pdf}
	\includegraphics[width=\linewidth]{images/comp_training_error_umls_small.pdf}			
	\includegraphics[width=\linewidth]{images/comp_training_error_nation_small.pdf}				
	\caption{\label{fig:r_vs_br} ROC-AUC scores of compositional models. The x-axis denotes the proportion of an observed triples including negative triples used for training models. We  use another 30\% of triples to evaluate the ROC-AUC score. In general, compositional models outperform BRESCAL and BLOGIT model when the size of training set is relatively small, whereas BRESCAL and BLOGIT perform slightly better than or comparable to the compositional models when the size of training set is relatively large. For UMLS dataset, the multiplicative compositional model consistently outperforms the other models across all training proportions.}
\end{figure} 


%!TEX root = ./cikm2016.tex
\section{Discussion}
We have proposed a novel compositional relational model with uncertainty and presented the
Thompson sampling for both compositional and non-compositional models to solve the active knowledge acquisition problem. 
The compositional model aims to infer the latent features of knowledge 
bases by incorporating an additional graph structure. In the passive 
learning scenario, the compositional model outperforms the other models, 
especially, when training size is relatively small. 
In the active learning scenario, Bayesian RESCAL achieves the highest 
cumulative gain across all datasets. Again, this result emphasise the 
importance of being balanced between exploration and exploitation. 

Previous work such as the one by \cite{kajino2015active} 
views knowledge population 
and predictive model construction are separate problems. 
We find this observation true when the algorithm has a warm-start, 
i.e. already having a fair amount of data before active learning starts; 
when the information is sparse, the same strategy works for both maximizing 
recall and reducing uncertainty.  
\eat{This finding
may hold for some scenarios where their is already a proper amount of 
information to exploit the structure we want to maximise, but the finding 
is not consistence in more conservative cases.}
Thompson sampling has been studied in the context of multi-armed bandit 
problems where the goal is to maximise cumulative gains or minimise cumulative 
regrets over time, whereas its performance on making a predictive model has not 
been widely discussed so far. Its performance on building a generalisable model 
was unclear. Throughout this work, we have empirically shown that maximising 
cumulative gain entails the predictive models as well.
In the long run, we see this work as a promising step towards using a composition-aware knowledge 
completion system to connect with the 
knowledge extraction problem~\cite{dong2014knowledge}. %, with rich graph structures. 

%\begin{description}
%  \item[Not useful theorem] In \cite{kawale2015efficient}, the theorem does not really give
%  a setting that is used in the paper.
%  \item[Logistic regression] For outputs which are binary, instead of a Gaussian assumption on
%  $x$, we can use logistic regression. See
%   \url{http://people.csail.mit.edu/romer/papers/TISTRespPredAds.pdf}
%  \item[Choosing triples] For entity $e_i$ related via relation $R_k$ to entity $e_j$, we want
%  an active learning algorithm that will choose triples $e_i R_k e_j$ that corresponds to
%  $x_{ikj}$. This will not have a bandit regret bound, since we only choose each triple once.
%  \item[Contextual Bandits] There are two groups of three possible assumptions for contextual
%  bandits:
%  \begin{enumerate}
%    \item Two elements of the tuple as context, choose only one element
%    \begin{itemize}
%      \item $e_iR_k$ context, $e_j$ arms
%      \item $e_i, e_j$ context, $R_k$ arms
%      \item $R_ke_j$ context, $e_i$ arms
%    \end{itemize}
%    \item One element of the tuple as context, choose two elements
%    \begin{itemize}
%      \item $e_i$ context, $R_k e_j$ arms
%      \item $R_k$ context, $e_i, e_j$ arms
%      \item $e_j$ context, $e_iR_k$ arms
%    \end{itemize}
%  \end{enumerate}
%\end{description}
%
%\section{Discussion}
%There are several limitation of this work. I'll discuss a few of them here, and suggest a new possible direction in a distance space.
%\begin{description}
%\item[Arbitrary output] Normal distribution with the binary output is an arbitrary choice. Why should it be normal? and binary? The problem is more arbitrary when it comes to compositional structure. We place the number of path as the output of compositional normal distribution, but there is no way to say that this approach works with the compositional structure we proposed.
%\item[Arbitrary compositionality] Even we proposed two compositional approaches, the choices of the compositional structures are also arbitrary and not interpretable.
%\end{description}
%
%Let's start from the scratch again. The most easiest way to interpret latent relation $R_k$ is to consider the latent relation as a linear transformation of latent entity $e_i$ to the transformed entity $e_i^\top R_k$, and this linear transformation $e_i^\top R_k$ should be somehow similar to $e_j$ when $(i, k, j)$ is a valid triple. This can be easily formulated in Euclidean distance space. We measure the distance between two entities as $|| e_i^\top R_k - e_j ||_2$ or $|| e_i - e_j^\top R_k^{-1} ||_2$, and let the distances of valid triples be shorter than those of invalid triples. We might want to model this relation in a probabilistic framework, and in this case, the first equation could be modelled as $e_j \sim \mathcal{N}_D(e_i^\top R_k, \Sigma)$ or the second equation as $e_i \sim \mathcal{N}_D(e_j^\top R_k^{-1}, \Sigma)$. The probability distribution of one entity is conditioned on the other entity, and we may model this as a Markov random field with the energy function between two entities as $E(e_i, e_j) = \exp(-||e_i^\top R_k - e_j||_2)$.
%
%Again, this formulation is somewhat weird and seems not well defined. Let's define it in the more general Mahalanobis distance space. In the Mahalanobis space, the distance between two entities $d_{R_k}(e_i, e_j)$ is $\sqrt{(e_i-e_j)^\top R_k (e_i - e_j)}$ (the corresponding probability distribution is a normal distribution with covariance matrix $R_k^{-1}$). 
%Now let's think about the compositional Mahalanobis space. Assume that triples $(i, k1, j)$ and $(j, k2, l)$ is valid triples and $(j, k2, l')$ is an invalid triple. If we define that the compositional Mahalanobis space has precision matrix $R_{k1,k2} = R_{k1} R_{k2}$, is it valid to say $d_{R_{k1,k2}}(i, l)$  of valid compositional path $(i, k1, j) - (j, k2, l)$ is shorter than $d_{R_{k1,k2}}(i, l')$ of invalid compositional path $(i, k1, j)-(j, k2, l')$. So the only thing that we need to prove is if $d_{R_{k1}}(i, j) + d_{R_{k2}}(j, l) < d_{R_{k1}}(i, j) + d_{R_{k2}}(j, l')$ then $d_{R_{k1,k2}}(i, l) < d_{R_{k1,k2}}(i, l')$?
%
%One problem with the distance approach is that the distance metric is symmetric, so we cannot encode the direction of relations. We may define asymmetric Mahalanobis space where the precision matrix $R$ is positive semi-definite but not symmetric.



%ACKNOWLEDGMENTS are optional
%\section{Acknowledgments}
%This section is optional; it is a location for you
%to acknowledge grants, funding, editing assistance and
%what have you.  In the present case, for example, the
%authors would like to thank Gerald Murray of ACM for
%his help in codifying this \textit{Author's Guide}
%and the \textbf{.cls} and \textbf{.tex} files that it describes.

%
% The following two commands are all you need in the
% initial runs of your .tex file to
% produce the bibliography for the citations in your paper.
\bibliographystyle{abbrv}
\bibliography{cikm2016}  % sigproc.bib is the name of the Bibliography in this case
% You must have a proper ".bib" file
%  and remember to run:
% latex bibtex latex latex
% to resolve all references
%
% ACM needs 'a single self-contained file'!
%
%APPENDICES are optional
%\balancecolumns
\appendix
%!TEX root = ./icml2016.tex
\section*{Posterior covariance of logistic output}
Let $R_k^*$ is a maximum a posterior solution of $R_k$ given $E$. Then, the conditional posterior covariance of relation matrix $R_k$ has the form of:

\begin{align}
S_i^{-1} = \sum_{x_{ikj}} \sigma(e_{i}^{\top} R_k^* e_{j}) (1 - \sigma(e_{i}^{*\top} R_k^* e_{j})) 
\bar{e}_{ij}\bar{e}_{ij}^\top + I\sigma_r^{-1}, \notag
\end{align}
where $\bar{e}_{ij} = e_i \otimes e_j$.

\section*{Rao-Blackwellisation particle Thompson sampling}
We also develop Rao-Blackwellisation particle Thompson sampling algorithm for the RESCAL with the Gaussian output model. The outline of the algorithm is described in Algorithm \ref{alg:rbsmc}. With the Rao-Blackwellisation, we marginalise out the relation matrix $R_k$ while computing the weight of each particle, but we still keep the same MCMC kernel to generate the samples. In theory, this will reduce the degeneracy problems for long running particles, but in our experiment, the difference between two models is not significant.
\begin{algorithm}[t!]
   \caption{Rao-Blackwellised Particle Thompson Sampling for Gaussian output}
   \label{alg:rbsmc}
\begin{algorithmic}
   \STATE {\bfseries Input:} $\sigma_x, \sigma_e, \sigma_r$.
   \FOR{$t=1,2, \dots$}
   \STATE \textit{Thompson Sampling}:
   \STATE $h_t \sim $ Cat$(\mathbf{w}^{t-1})$
   \STATE $(i,k,j) \leftarrow \arg\max p(x_{ikj}| E^{h_t})$   % \hfill $\triangleright$ see Table. (\ref{tab:brescalposterior})
   \STATE Query $(i,k,j)$ and observe $x_{ikj}$
   \STATE $\mathcal{X}^{t} \leftarrow \mathcal{X}^{t-1} \cup x_{ikj}$

   \STATE \textit{Particle Filtering}:
   \STATE $\forall h, w_h^{t} \propto p(x_{ikj} | E^{h_t})$   \hfill $\triangleright$ Reweighting%, Table. (\ref{tab:brescalposterior})
   \IF{ESS$(\mathbf{w}^{t}) \leq N$}
   \STATE resample particles
   \STATE $w_h^{t} \leftarrow 1/H$
   \ENDIF

   \FOR{$h=1$ {\bfseries to} $H$}
   \STATE $\forall k, R_k^{h} \sim p(R_k | \mathcal{X}^{t}, E^{h_{t-1}})$   \hfill $\triangleright$ Auxiliary sampling%, see Table. (\ref{tab:brescalposterior})
   \STATE $\forall i, e^{h}_i \sim p(e_i | \mathcal{X}^{t}, E^{h}_{-i}, \mathcal{R}^{h_t})$ %\hfill $\triangleright$ see Table. (\ref{tab:brescalposterior})
   \ENDFOR

   \ENDFOR
\end{algorithmic}
\end{algorithm}

%!TEX root = ./cikm2016.tex
\section{Posterior Distribution of Compositional Relations}

We provide the conditional posterior distributions of the compositional models.

\subsection{Additive Compositionality}

The conditional distribution of $e_i$ given $E_{-i}, \mathcal{R}, \mathcal{X}^{t}, \mathcal{X}^{L(t)}$ is
expanded from the posterior of PRESCAL by incorporating compositional triples.
\begin{align} \label{eqn:comp_sample_e}
p(e_i |E_{-i}, \mathcal{R}, \mathcal{X}^{t}, \mathcal{X}^{L(t)}) &= \mathcal{N}(e_i | \mu_i, \Lambda_i^{-1}),
\end{align}
where
\begin{align*}
\mu_i &= \Lambda_i^{-1}\xi_i \\
\Lambda_i &= \frac{1}{\sigma_x^2} \sum_{jk : x_{ikj} \in \mathcal{X}^{t}} (R_k e_j)(R_k e_j)^\top \\
&\quad+ \frac{1}{\sigma_x^2} \sum_{jk : x_{jki} \in \mathcal{X}^{t}} (R_k^\top e_j)(R_k^\top e_j)^\top \\
&\quad + \frac{1}{\sigma_c^2} \sum_{jc : x_{icj} \in \mathcal{X}^{L(t)}} (R_c e_j)(R_c e_j)^\top \\
&\quad+ \frac{1}{\sigma_c^2} \sum_{jc : x_{jci} \in \mathcal{X}^{L(t)}} (R_c^\top e_j)(R_c^\top e_j)^\top +
\frac{1}{\sigma_e^2} {I}_D \\
\xi_i &= \frac{1}{\sigma_x^2}\sum_{jk : x_{ikj} \in \mathcal{X}^{t}}  x_{ikj} R_{k} e_{j} + \frac{1}
{\sigma_x^2}\sum_{jk : x_{jki} \in \mathcal{X}^{t}} x_{jki} R_{k}^\top e_{j} \\
& + \frac{1}{\sigma_c^2}\sum_{jc : x_{icj} \in \mathcal{X}^{L(t)}}  x_{icj} R_{c} e_{j} + \frac{1}
{\sigma_c^2}\sum_{jc : x_{jci} \in \mathcal{X}^{L(t)}} x_{jci} R_{c}^\top e_{j}
\end{align*}

To compute the conditional distribution of $R_k$, we first decompose $R_c$ into two part where $R_c =
\frac{1}{|c|} R_k + \frac{|c|-1}{|c|}R_{c/k}$, where $R_{c/k} = \sum_{k' \in c/k} R_{k'}$.
The distribution of compositional triple is decomposed as follows:
\begin{align}
x_{(i, c, l)} \sim \mathcal{N}(e_i^\top (\frac{1}{|c|} R_k + \frac{|c|-1}{|c|}R_{c/k}) e_j, \sigma_{c}^2).
\end{align}
Then, the conditional distribution $R_k$ given $R_{-k}, E, \mathcal{X}^{t}, \mathcal{X}^{L(t)}$ is
\begin{align}
\label{eqn:comp_cond_r}
p(R_k|E, \mathcal{X}^{t}, \mathcal{X}^{L(t)}, \sigma_r, \sigma_x)  &= \mathcal{N}(\text{vec}(R_k) | \mu_k,
\Lambda_k^{-1}),
\end{align}
where
\begin{align*}
\mu_k &=\Lambda_k^{-1}\xi_k \\
\Lambda_k &= \frac{1}{\sigma_x^2} \sum_{ij:x_{ikj} \in \mathcal{X}^{t}} \bar{e}_{ij}\bar{e}_{ij}^\top + \frac{1}
{\sigma_r^2} {I}_{D^2} \\
& +\frac{1}{|c|^2 \sigma_c^2} \sum_{ij:x_{icj} \in \mathcal{X}^{L(t)},\text{ }k \in c} \bar{e}_{ij} \bar{e}_{ij}^\top \\
\xi_k &=  \frac{1}{\sigma_x^2}\sum_{ij:x_{ikj} \in \mathcal{X}^{t}} x_{ikj} \bar{e}_{ij}\\
& +\frac{1}{|c| \sigma_c^2} \sum_{ij:x_{icj} \in \mathcal{X}^{L(t)},\text{ }k \in c} x_{icj} \bar{e}_{ij} - \frac{|c|-1}{|c|}
\bar{e}_{ij} r_{c/k}^\top \bar{e}_{ij}\\
\bar{e}_{ij} &= e_{i} \otimes e_{j}.
\end{align*}
Vectorisation of $R_c$ and $R_{c/k}$ are represented as $r_c$ and $r_{c/k}$, respectively.

The detail derivation of the posterior distribution is as follows:
\begin{align*}
&p(R_k | E, R_{-k}, \mathcal{X}) \propto p(\mathcal{X} | R, E)p(R_k)&\\
&\propto \prod_{x_{ikj}}\exp\bigg\{-\frac{(x_{ikj} - e_i^\top R_k e_j)^2}{2\sigma_x^2}\bigg\} &\\
& \quad\prod_{x_{icj}} \exp\bigg\{-\frac{(x_{icj} - e_i^\top R_c e_j)^2}{2\sigma_c^2}\bigg\} \exp\bigg\{-\frac{r_k^\top r_k}{2\sigma_r^2}\bigg\}&\\
&= \exp\bigg\{-\frac{\sum_{x_{ikj}}(x_{ikj} - \bar{e}_{ij}^\top r_k)^2}{2\sigma_x^2} &\\
&\quad- \frac{\sum_{x_{icj}}(x_{icj} - \bar{e}_{ij}^\top r_c)^2}{2\sigma_c^2} -\frac{r_k^\top r_k}{2\sigma_r^2} \bigg\}&\\
&= \exp\bigg\{-\frac{\sum_{x_{ikj}}(x_{ikj} - \bar{e}_{ij}^\top r_k)^2}{2\sigma_x^2} &\\ 
&\quad-\frac{\sum_{x_{icj}}(x_{icj} - \bar{e}_{ij}^\top (\frac{1}{|c|}r_k + \frac{|c|-1}{|c|})r_{c/k})^2}{2\sigma_c^2} -\frac{r_k^\top r_k}{2\sigma_r^2} \bigg\}&\\
&= \exp\bigg\{ -\frac{\sum_{x_{ikj}}- 2 x_{ikj} \bar{e}_{ij}^\top r_k + r_k^\top \bar{e}_{ij} \bar{e}_{ij}^\top r_k }{2\sigma_x^2} &\\
&\quad-\frac{\sum_{x_{icj}} \frac{2}{|c|} r_k ^\top (x_{icj} - \frac{(|c|-1)}{|c|} \bar{e}_{ij}^\top r_{c/k}) + (\frac{1}{|c|^2}r_k^\top \bar{e}_{ij} \bar{e}_{ij}^\top r_k)}{2\sigma_c^2} &\\
&\quad-\frac{r_k^\top r_k}{2\sigma_r^2} + const \bigg\}&\\
&\propto \exp\bigg\{ - \frac{1}{2}r_k^\top(\frac{1}{\sigma_x^2}\sum_{x_{ikj}} \bar{e}_{ij}\bar{e}_{ij}^\top + \frac{1}{|c|^2\sigma_c^2}\sum_{x_{icj}} \bar{e}_{ij}\bar{e}_{ij}^\top + \frac{1}{\sigma_r^2}I) r_k  &\\
&- r_k^\top \Big(\frac{\sum_{x_{ikj}}-x_{ikj}\bar{e}_{ij}}{\sigma_x^2} + \frac{\sum_{x_{icj}} |c|^{-1} (x_{icj} - \frac{(|c|-1)}{|c|} \bar{e}_{ij}^\top r_{c/k})}{\sigma_c^2} \Big)&
\end{align*}
Completing the square results Equation \ref{eqn:comp_cond_r}.

%Note that the ordering of the relations in compositional sequence $c$ does not affect the value of
%compositional triple $(i, c, j)$.

\subsection{Multiplicative Compositionality}

Given a sequence of relations including relation $k$, $R_k$ is placed in the middle of the compositional
sequence, i.e., $e_i^\top R_{c(1)}R_{c(2)} \dots R_{c(\delta_k)} \dots R_{c(|c|-1)}R_{c(|c|)} e_j$, where $
\delta_k$ is the index of relation $k$. For notational simplicity, we will denote the left side $e_i^\top R_{c(1)}
R_{c(2)} \dots R_{c(\delta_k -1)}$ as $\bar{e}_{ic(:\delta_k)}^\top$, and the right side $R_{c(\delta_k + 1)} \dots
R_{c(|c|-1)}R_{c(|c|)} e_j$ as $\bar{e}_{ic(\delta_k:)}$, therefore we can rewrite the mean parameter as $
\bar{e}_{ic(:\delta_k)}^\top R_{k} \bar{e}_{ic(\delta_k:)}$. With the simplified notations, the conditional of $R_k$
is
\begin{align}
p(R_k|E, \mathcal{X}, \sigma_r, \sigma_x)  &= \mathcal{N}(\text{vec}(R_k) | \mu_k, \Lambda_k^{-1}),
\end{align}
where
\begin{align*}
\mu_k &= \Lambda_k^{-1}\xi_k \\
\Lambda_k &= \frac{1}{\sigma_x^2} \sum_{ij:x_{ikj} \in \mathcal{X}^{t}} (e_i \otimes e_j)(e_i \otimes e_j)^\top +
\frac{1}{\sigma_r^2} {I}_{D^2} \\
+ &\frac{1}{\sigma_c^2} \sum_{ij:x_{icj} \in \mathcal{X}^{L(t)}, \text{ }k \in c} (\bar{e}_{ic(:\delta_k)} \otimes
\bar{e}_{jc(\delta_k:)})(\bar{e}_{ic(:\delta_k)} \otimes \bar{e}_{jc(\delta_k:)} )^\top \\
\xi_k &= \frac{1}{\sigma_x^2} \sum_{ij:x_{ikj} \in \mathcal{X}^{t}} x_{ikj} (e_{j} \otimes e_{i}) \\
& + \frac{1}{\sigma_c^2} \sum_{ij:x_{icj} \in \mathcal{X}^{L(t)}, \text{ }k\in c} x_{icj} (\bar{e}_{ic(:\delta_k)}
\otimes \bar{e}_{jc(\delta_k:)}).
\end{align*}
The conditional distribution of $e_i$ given the rest is the same as the additive compositional case.

%If the determinant of latent relation $R_k$ is greater than 1, the compositional latent relation $R_c$ might
%be exploded after multiplying a long sequence of relations. To obtain a stable scale of compositional
%relation $R_c$, one may multiply decaying factor $\tau < 1$ after each composition. $R_c = \tau^{|c|-1}
%R_{c(1)} R_{c(2)} \dots R_{c(|c|)}$.

%%Appendix A
%\section{Headings in Appendices}
%The rules about hierarchical headings discussed above for
%the body of the article are different in the appendices.
%In the \textbf{appendix} environment, the command
%\textbf{section} is used to
%indicate the start of each Appendix, with alphabetic order
%designation (i.e. the first is A, the second B, etc.) and
%a title (if you include one).  So, if you need
%hierarchical structure
%\textit{within} an Appendix, start with \textbf{subsection} as the
%highest level. Here is an outline of the body of this
%document in Appendix-appropriate form:
%\subsection{Introduction}
%\subsection{The Body of the Paper}
%\subsubsection{Type Changes and  Special Characters}
%\subsubsection{Math Equations}
%\paragraph{Inline (In-text) Equations}
%\paragraph{Display Equations}
%\subsubsection{Citations}
%\subsubsection{Tables}
%\subsubsection{Figures}
%\subsubsection{Theorem-like Constructs}
%\subsubsection*{A Caveat for the \TeX\ Expert}
%\subsection{Conclusions}
%\subsection{Acknowledgments}
%\subsection{Additional Authors}
%This section is inserted by \LaTeX; you do not insert it.
%You just add the names and information in the
%\texttt{{\char'134}additionalauthors} command at the start
%of the document.
%\subsection{References}
%Generated by bibtex from your ~.bib file.  Run latex,
%then bibtex, then latex twice (to resolve references)
%to create the ~.bbl file.  Insert that ~.bbl file into
%the .tex source file and comment out
%the command \texttt{{\char'134}thebibliography}.
%% This next section command marks the start of
%% Appendix B, and does not continue the present hierarchy
%\section{More Help for the Hardy}
%The sig-alternate.cls file itself is chock-full of succinct
%and helpful comments.  If you consider yourself a moderately
%experienced to expert user of \LaTeX, you may find reading
%it useful but please remember not to change it.
%%\balancecolumns % GM June 2007
%% That's all folks!

\end{document}
