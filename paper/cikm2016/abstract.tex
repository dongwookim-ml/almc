%!TEX root = ./cikm2016.tex
\begin{abstract}

Knowledge base construction consists of two tasks: extracting information from external sources  (knowledge population), and inferring missing information through a statistical analysis on the extracted information (knowledge completion).
In many cases, there are not enough external sources to extract information for a comprehensive
knowledge base\eat{, hence the need for a separate completion step}.
An incremental knowledge population approach via labelling of human experts can help to reduce the gap between these two processes.
We propose a new probabilistic knowledge base factorisation method that benefits from the path structure of existing knowledge (e.g. syllogism). Our method enables a common computation
and modelling approach to be used for both knowledge base construction tasks.
Empirical experiments show that our model improves over the non-probabilistic and non-path counterparts on the knowledge completion task.
The probabilistic formulation allows us to develop an incremental knowledge population model that trades off exploitation and exploration.
%We demonstrate that the factorisation model with the path structure performs better on the knowledge completion task.
We show that our proposed approach for incremental knowledge population performs significantly
better than sampling knowledge triples at random.
\eat{
However the path structure does not seem
to help in this setting.
In applications where there is a continual improvement in knowledge bases,
our common probabilistic model bridges the gap between knowledge population and knowledge
completion.
}

\eat{
Our incremental model performs better on empirical benchmarks than
Whereas the model without the path structure performs better in the incremental population.
The result leads to a counter-intuitive conclusion; a better predictive model does not guarantee to have a better performance in incremental population.
An additional experiment explains the degeneracy under the model uncertainty.
}
\eat{
%Automated processes for knowledge base completion benefit from models that account for
%statistical uncertainty.
We are concerned with the problem of knowledge base completion, or inferring missing facts from known relations.
%  It is desirable to take into account paths and compositions, and
%  able to connect to the knowledge extraction problem by actively
%  acquiring labeled data points.
%  Recent knowledge completion algorithms could do either, but not both.
A human can readily infer a new fact through the composition of known facts in knowledge base,
whereas the current statistical relational models lack the consideration of
active knowledge acquisition through the composition of knowledge.
We propose a probabilistic model to bridge this gap.
We start from a new formulation of vector space embedding for knowledge tensors  %which generalises the RESCAL model to
that explicitly model distributions of relations. This
enables us to extend Thompson sampling approach to knowledge bases, and
%we demonstrate the benefit of active knowledge acquisition. We
also incorporate additive and multiplicative
approaches for composing relations.
On synthetic and real world datasets,
we find that learning with composition is helpful
when training data is sparse,
and that Thompson sampling provides effective exploitation-exploration strategies
that balance recall and reconstruction accuracy.
% and show the regimes where compositional models are
%beneficial for knowledge base completion.
}
\end{abstract}
