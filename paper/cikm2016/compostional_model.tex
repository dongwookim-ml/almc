%!TEX root = ./cikm2016.tex
\section{Compositional Relations}
\label{sec:comp}

In this section, we propose a compositional relation model that exploits the compositional structure of
knowledge graphs to capture the latent semantic structure of the entities and relations.
A very recent study shows the benefit of using compositionality in the vector space model~\cite{guu2015traversing}. Here, we further extend their framework in a probabilistic way.
%While previously suggested vector space models provide a statistical way to infer the latent semantic
%structure of entities and relations, but lack consideration of a graph structure of a knowledge graph itself.
%%LX: the above para is also in intro right now.

The compositionality represents a semantic meaning of a path over a knowledge graph that corresponds to a
sequence of composable triples.
For example, given two triples, ``Barack Obama is a 44th president of U.S.'' \texttt{(BarackObama, PresidentOf,
U.S)} and ``Joe Biden was a running mate of Barack Obama'' \texttt{(JoeBiden, RunningMateOf, BarackObama)},
one can naturally deduce that the ``Joe Biden is a vice president of U.S.'' \texttt{(JoeBiden, VicePresidentOf, U.S.)}.
Here the composition of two relations, president of, and running mate of, yields a compositional relation,
vice president of.
% Is every running mate always a vice president? maybe this is not a proper example.
More formally, if there is a sequence of triples where the target entity of a former triple is a source entity of a
latter triple in a consecutive pair of triples in the sequence, then we can form a compositional triple
as follows.
Given the sequence of $n$ triples
$(i_1, k_1 ,j_1)$,  $(i_2, k_2, j_2)$, $(i_3, k_3, j_3)$ $\dots$ $(i_n, k_n, j_n)$, where $j_{k}=i_{k+1}$ for all $k$,  we form a compositional triple $(i_1, {c}(k_1, k_2, \dots, k_n), j_n)$, where $c$ denotes the compositional
relation of the sequence of relations.

Let $\mathcal{C}^{L}$ be a set of all possible compositions whose length is up to $L$, $c \in \mathcal{C}$
be a sequence of relations, $c(i)$ be $i$th index of a relation in sequence $c$ and $|c|$ be the length of the
sequence. With set of compositions $\mathcal{C}^{L}$, we can expand set of observed triples
$\mathcal{X}^{t}$ to set of compositional triples $\mathcal{X}^{\mathcal{C}^{L}(t)}$ in which
compositional triple $x_{icj}$ is an
indicator variable that show the existence of the path from entity $i$ to entity $j$ through sequence
of relations
$c$ in $\mathcal{X}^{t}$. Note that the compositional relation $c$ is an abstract relation, and there might be a
multiple possible paths from entity $i_1$ to $j_n$.

With these extended compositional triples, we again model $x_{icj}$ with a bilinear Gaussian distribution,
\begin{align}
x_{(i, {{c}(k_1, k_2)}, l)} \sim \mathcal{N}(e_i^\top R_{{c}(k_1,k_2)} e_j, \sigma_{c}^2),
\end{align}
where $R_{{c}(k_1,k_2)} \in \mathbb{R}^{D\times D}$ is a latent matrix of compositional relation $c$, and $
\sigma_{c}^2$ is a covariance of the compositional triples. Again the entity vectors are shared across the compositional and non-compositional triples.
In the subsequent sections, we provide two different ways of modelling the compositional relation $R_c$.

\subsection{Additive Compositionality}
With the compositions of relations, the \textsc{Prescal} may place a different relation matrix $R_c$ for each composition $c$. However the number of required matrices increases exponentially with respect to the length of composition. Consequently, the computational cost will also increase exponentially.
To limit the required number of parameters, we propose two different ways of modelling the compositional relation $R_c$.
First, we define an additive compositional relation $R_c$ as a normalised sum over
the sequence of relation matrices in composition $c$, i.e.,
$R_{{c}} = \frac{1}{|c|}(R_{c(1)} + R_{c(2)} + \dots + R_{c(|c|)})$, then compositional triple $x_{icj}$
is modelled as
\begin{align}
x_{(i, c, j)} &\sim \mathcal{N}(e_i^\top R_c e_j, \sigma_{c}^2) \\
&= \mathcal{N}(e_i^\top \frac{1}{|c|}(R_{c(1)} + R_{c(2)} + \dots + R_{c(|c|)}) e_j, \sigma_{c}^2). \notag
\end{align}
The conditional distribution of $e_i$ and $R_k$ given the rest can be obtained in the same way used for the posterior distribution of \textsc{Prescal}. The details of the parameter estimators are shown in the appendix.

\eat{
given $E_{-i}, \mathcal{R}, \mathcal{X}^{t}, \mathcal{X}^{L(t)}$ is
expanded from the posterior of \textsc{Prescal} by incorporating compositional triples.
\begin{align} \label{eqn:comp_sample_e}
p(e_i |E_{-i}, \mathcal{R}, \mathcal{X}^{t}, \mathcal{X}^{L(t)}) &= \mathcal{N}(e_i | \mu_i, \Lambda_i^{-1}).
\end{align}

To compute the conditional distribution of $R_k$, we first decompose $R_c$ into two part where $R_c =
\frac{1}{|c|} R_k + \frac{|c|-1}{|c|}R_{c/k}$, where $R_{c/k} = \sum_{k' \in c/k} R_{k'}$.
The distribution of compositional triple is decomposed as follows:
\begin{align}
x_{(i, c, l)} \sim \mathcal{N}(e_i^\top (\frac{1}{|c|} R_k + \frac{|c|-1}{|c|}R_{c/k}) e_j, \sigma_{c}^2).
\end{align}
Then, the conditional distribution $R_k$ given $R_{-k}, E, \mathcal{X}^{t}, \mathcal{X}^{L(t)}$ is
\begin{align}
\label{eqn:comp_cond_r}
p(R_k|E, \mathcal{X}^{t}, \mathcal{X}^{L(t)}, \sigma_r, \sigma_x)  &= \mathcal{N}(\text{vec}(R_k) | \mu_k,
\Lambda_k^{-1}).
\end{align}

The mean and precision are obtained by expanding out the sum across $\mathcal{X}^{t}$ and
$\mathcal{X}^{L(t)}$. The details of the parameters are shown in the appendix.
}

\subsection{Multiplicative Compositionality}
Second, we define an multiplicative compositional relation $R_c$ as a sequence of multiplication over
relations in composition $c$, i.e. $R_c = R_{c(1)} R_{c(2)} \dots R_{c(|c|)}$, and the compositional triple as a
bilinear Gaussian distribution with the compositional relation $R_c$,
\begin{align}
\label{eqn:multi}
x_{(i, c, j)} \sim \mathcal{N}(e_i^\top R_{c(1)}R_{c(2)} \dots R_{c(|c|-1)}R_{c(|c|)} e_j, \sigma_{c}^2)
\end{align}
The multiplicative compositionality can be understood as a sequence of linear transformation from the original
entity $i$ with the compositional relations, and the inner product between the transformed entity and target
entity forms a value of the compositional triple. Again, the details of the parameter estimators are shown in the appendix.

\eat{
Given a sequence of relations including relation $k$, $R_k$ is placed in the middle of the compositional
sequence, i.e., $e_i^\top R_{c(1)}R_{c(2)} \dots R_{c(\delta_k)} \dots R_{c(|c|-1)}R_{c(|c|)} e_j$, where $
\delta_k$ is the index of relation $k$. For notational simplicity, we will denote the left side $e_i^\top R_{c(1)}
R_{c(2)} \dots R_{c(\delta_k -1)}$ as $\bar{e}_{ic(:\delta_k)}^\top$, and the right side $R_{c(\delta_k + 1)} \dots
R_{c(|c|-1)}R_{c(|c|)} e_j$ as $\bar{e}_{ic(\delta_k:)}$. With this notation, we can rewrite the mean parameter as $
\bar{e}_{ic(:\delta_k)}^\top R_{k} \bar{e}_{ic(\delta_k:)}$, and the conditional of $R_k$
as
\begin{align}
p(R_k|E, \mathcal{X}, \sigma_r, \sigma_x)  &= \mathcal{N}(\text{vec}(R_k) | \mu_k, \Lambda_k^{-1}),
\end{align}
where
the mean and precision are obtained by expanding out the product across $\mathcal{X}^{t}$ and
$\mathcal{X}^{L(t)}$. The details of the parameters are shown in the appendix.
The conditional distribution of $e_i$ given the rest is the same as Equation $\ref{eqn:comp_sample_e}$.
}

As the length of sequence $c$ increases, a small error in the first few multiplications will result a large
differences in the final compositional relation. One way to mitigate the cascading error is to increase variance
of compositional triples $\sigma_c^2$ as the length of the sequence increases.

%If the determinant of latent relation $R_k$ is greater than 1, the compositional latent relation $R_c$ might
%be exploded after multiplying a long sequence of relations. To obtain a stable scale of compositional
%relation $R_c$, one may multiply decaying factor $\tau < 1$ after each composition. $R_c = \tau^{|c|-1}
%R_{c(1)} R_{c(2)} \dots R_{c(|c|)}$.
